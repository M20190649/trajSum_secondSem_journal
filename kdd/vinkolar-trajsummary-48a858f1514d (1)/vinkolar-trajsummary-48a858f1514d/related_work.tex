\section{Related Work}
\label{sec:related}
Trajectory distance metrics and trajectory clustering for generic trajectories have been studied widely over the past years. There have been various papers proposing new distance measures and clustering mechanisms. We now describe and differentiate the related studies in this area.

\paragraph{Trajectory Distance}
The most commonly used similarity measure $L^p$-Norm metric. The Euclidean metric is used to measure similarity, coupled with Discrete Fourier transform to reduce the dimensionality and R-tree was used for indexing~\cite{lp1}. Various Fast and efficient methods for indexing and retrieval of similar series, when the similarity is defined by the Euclidean similarity are proposed in \cite{lp1},\cite{lp2}. 

The main problem with Euclidean distance is its inability to handle noise and local time shifting. In order to overcome this, exploration in other techniques began. Berndt \emph{et. al.} \cite{dtw} introduced Dynamic Time Warping which allowed  a time series to be stretched so as to match the query series. In variations to DTW, Vlachos \emph{et al.} \cite{Vlachos2002} applied Longest Common Subsequence(LCSS) measure, and Chen \emph{et al.}\cite{Chen2005} applied  Edit Distances on Real Sequences(EDR). But all these measures, i.e., DTW, LCSS , and EDR are not metrics and don not follow triangle inequality. As an improvement, Chen \emph{et al.}\cite{Chen2005} introduced Edit Distance with Real Penalty (ERP) which handled local time shifting and is a metric. Wang \emph{et. al.} compare six widely used similarity measures including measures such as Euclidean distance, DTW, ERP, EDR and LCSS and their performances as trajectory similarity measures~\cite{simcomp}. In contrast to exploring generic trajectory distance metrics, we evaluate the existing metrics to show which metric provides the right abstraction for summarizing user trips.

\paragraph{Trajectory Clustering}
Gaffney \emph{et. al.} study trajectory clustering using the Expectation Maximization algorithm~\cite{gaffney1999trajectory}. Nanni \emph{et. al.} proposes an adaptation of a density-based clustering algorithm for trajectories~\cite{nanni2006time}. Zhang \emph{et. al.} uses kernel density estimation based approach for the clustering of spatio-temporal trajectories~\cite{Zhang}. Network-level trajectory clustering is studied by clustering trajectories after snapping the trajectories on some maps~\cite{mapmatch1, mapmatch2}. While the complexity of the clustering scheme reduces due to constrained snapping of trajectories to a discrete graph space, existing map-matching techniques utilize time-consuming approaches such as Hidden Markov Model~\cite{krumm}. Other studies summarize trips by creating text summaries of trips using a partition and summarization approach~\cite{su2015making}. 

Li \emph{et al.}\cite{Li2010} study moving clusters and detecting closed swarms for trajectories. Moving clusters in trajectory clustering is also studied in other existing work~\cite{flock1},\cite{flock2}. Lee \emph{et. al} proposes TRA-CLUS, a partition and grouping framework for clustering trajectories~\cite{Lee2007}. We compare our studies to the existing approaches, and show that none of the current approaches can be directly applied to summarizing user trips.
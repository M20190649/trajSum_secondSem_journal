\begin{abstract}
Mobility data of people is being increasingly recorded by location sensing applications such as GPS traces and Cellular Network Records. Such large-scale location data of people is capable of providing rich mobility context information about how, where and when an individual moves. These insights are useful in several domains such as hyper-targeted advertising, city transportation planning and cellular network planning. While mobility data is capable of providing such interesting information, techniques to summarize a spatio-temporal mobility of a individual is non-existent. In this work, we propose an "`Mobility Summary (MoSum)", a system for summarizing and quantifying how an individual moves in space and time. MoSum provides a novel way to store mobility signatures of a person, which inturn provides a powerful mechanism for solving various use-cases that can rely on regular movement pattern of an individual (such as next-location prediction and anomalous movement detection). %We show that our algorithm is -- at a median -- 20\% better and 5x faster than the trajectory clustering algorithms.
\end{abstract}